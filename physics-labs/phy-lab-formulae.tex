\documentclass{article}
\usepackage[utf8]{inputenc}
\usepackage[landscape]{geometry}
\usepackage{url}
\usepackage{multicol}
\usepackage{amsmath}
\usepackage{esint}
\usepackage{amsfonts}
\usepackage{tikz}
\usetikzlibrary{decorations.pathmorphing}
\usepackage{amsmath,amssymb}

\usepackage{colortbl}
\usepackage{ tipa }
\usepackage{xcolor}
\usepackage{mathtools}
\usepackage{amsmath,amssymb}
\usepackage{enumitem}
\makeatletter

\newcommand*\bigcdot{\mathpalette\bigcdot@{.5}}
\newcommand*\bigcdot@[2]{\mathbin{\vcenter{\hbox{\scalebox{#2}{$\m@th#1\bullet$}}}}}
\makeatother

\title{130 Cheat Sheet}
\usepackage[brazilian]{babel}
\usepackage[utf8]{inputenc}
\advance\topmargin-.8in
\advance\textheight3in
\advance\textwidth3in
\advance\oddsidemargin-1.5in
\advance\evensidemargin-1.5in
\parindent0pt
\parskip2pt
\newcommand{\hr}{\centerline{\rule{3.5in}{1pt}}}
%\colorbox[HTML]{e4e4e4}{\makebox[\textwidth-2\fboxsep][l]{texto}
\begin{document}

\begin{center}{\huge{\textbf{Phy F110}}}\\
\end{center}
\begin{multicols*}{3}

\tikzstyle{mybox} = [draw=black, fill=white, very thick,
    rectangle, rounded corners, inner sep=10pt, inner ysep=10pt]
\tikzstyle{fancytitle} =[fill=black, text=white, font=\bfseries]

%Errors
\begin{tikzpicture}
\node [mybox] (box){%
    \begin{minipage}{0.3\textwidth}
    For $N$ measurements of a quantity $Q$:\\
    $\overline{Q} = \frac{1}{N}\sum Q_i$\\
    Deviation $d$ = $\sqrt{\frac{1}{N}\sum\left(Q_i-\overline{Q}\right)^2}$\\
    Resultant measurement $Q = \overline{Q} \pm d$\\
    Error $\Delta Q$ = \textbf{Standard error} = $d$\\
    For a quantity $Q = Q(x,y,z)$, the general error is given as:\\
    \begin{equation*}
        \left(\Delta Q\right)^2 =  \left(\Delta Q_x\right)^2 +  \left(\Delta Q_y^2\right)^2 +  \left(\Delta Q_z^2\right)^2
    \end{equation*}
    Where $\Delta Q_x$ = $\left(\frac{\partial Q}{\partial x}\right)\Delta x$ and so on for $Q_y$ and $Q_z$\\
    Combination of errors:\\
    \\
	\begin{tabular}{lp{6cm} l}
    	$Q=x\pm y$ & $\Delta Q = \sqrt{\left(\Delta x\right)^2+\left(\Delta x\right)^2}$\\
    	$Q=xy$ or $\frac{x}{y}$ & $\left(\frac{\Delta Q}{\overline{Q}}\right)=\sqrt{\left(\frac{\Delta x}{\overline{x}}\right)^2+\left(\frac{\Delta y}{\overline{y}}\right)^2}$\\
    	$Q=x^n$ & $\left(\frac{\Delta Q}{\overline{Q}}\right)=n\frac{\Delta x}{\overline{x}}$\\
    	$Q=\ln{x}$ & $\Delta Q=\frac{\Delta x}{\overline{x}} $\\
    	$Q=e^x$ & $\frac{\Delta Q}{\overline{Q}}=\Delta x$\\
    \end{tabular}
    \end{minipage}
};

\node[fancytitle, right=10pt] at (box.north west) {Errors};
\end{tikzpicture}

%Planck's constant
\begin{tikzpicture}
\node [mybox] (box){%
    \begin{minipage}{0.3\textwidth}
    Boltzmann constant ($k$) = $1.38\cdot10^{-23}J/K$\\
    Planck's constant = $h$\\
    $U(\nu)d\nu = \frac{8\pi h\nu^3}{c^3}\frac{1}{\left(e^{\frac{h\nu}{kT}}-1\right)}d\nu$ (now, all relevant approximations can be made)\\
    At a particular frequency $\nu$, the photocurrent $I_{\text{ph}}$ is approximated to be:
    \begin{equation*}
        I_{\text{ph}} = \frac{8\pi h\nu^3}{c^3}e^{-\frac{h\nu}{kT}}
    \end{equation*}
    If we take the \textit{natural logarithm} on both sides, the following is obtained:
    \begin{equation*}
        \ln{I_{\text{ph}}} = -\frac{h\nu}{kT} + \text{constant}
    \end{equation*}
    Temperature dependence of resistance of tungsten filament:
    \begin{equation*}
        R = R_0(1+\alpha T+\beta T^2)
    \end{equation*}
    $\alpha, \beta$ being empirical constants used for calibration
    \end{minipage}
};

\node[fancytitle, right=10pt] at (box.north west) {Planck's constant};
\end{tikzpicture}

%EMI
\begin{tikzpicture}
\node [mybox] (box){%
    \begin{minipage}{0.3\textwidth}
    \textepsilon =$-\frac{d\phi}{dt}$\\
    Where \textepsilon = EMF and $\phi$ = flux through the coil\\
    $\omega_{\text{max}}=2\sqrt{\frac{Mgl}{I}}\sin{\frac{\theta_0}{2}}$\\
    Here, $\sqrt{\frac{Mgl}{I}}$ is the \textit{frequency} of oscillations, which can also be used to find the time period $T$\\
    Considering $R$ to be the radius of the arc (and $v_{\text{max}}$ to be the maximum velocity of the arc), we get:\\
    \begin{equation*}
        v_{\text{max}} = R\omega_{\text{max}} = \frac{4\pi R}{T}\sin{\frac{\theta_0}{2}}
    \end{equation*}
    After a bunch of approximations, it is concluded that \textepsilon$_{\text{max}}$ $\propto$ $v_{\text{max}}$ which is given a bit more precisely as:\\
    \begin{center}
        \textepsilon$_{\text{max}} = -\frac{1}{R}\frac{d\phi}{d\theta}\Big|_{\theta_{\text{max}}}v_{\text{max}} $
    \end{center}
    Consider the pulse width of one oscillation to be $\tau$ and the time constant of the circuit to be $RC$. If:
    \begin{itemize}
        \item $RC<\tau\implies$ Capacitor fully charged in one oscillation
        \item $RC>\tau\implies$ Capacitor charges in multiple oscillations
    \end{itemize}
    Total charge delivered to capacitor after each swing = $q$ = $\frac{\Delta \phi}{R}$\\
    For a damped oscillation, the angular displacement is given as:
    \begin{equation*}
        \theta_A(t) = \theta_{A_0}e^{-\frac{\omega_0t}{2Q}}
    \end{equation*}
    $Q$ being the \textbf{Quality factor} or strength of damping\\
    $Q \propto \frac{1}{\text{Amount of damping}}$\\
    In a plot of $\ln{\theta_{An}}$ vs. $n$, the equation is found to be:
    \begin{equation*}
        \ln{\theta_{An}} = \ln{\theta_{A0}} - \frac{\pi}{Q}n
    \end{equation*}
    $n$ being the number of oscillations
    \end{minipage}
};

\node[fancytitle, right=10pt] at (box.north west) {EMI};
\end{tikzpicture}

%Newton's rings
\begin{tikzpicture}
\node [mybox] (box){%
    \begin{minipage}{0.3\textwidth}
    Optical path difference for interfering waves = $2\mu t$\\
    Conditions for:
    \begin{enumerate}
        \item Constructive interference: $2t = m\lambda$
        \item Destructive interference: $2t = m\left(\lambda+\frac{1}{2}\right)$
    \end{enumerate}
    Where $m \in \mathbb{Z^{+}}\cup 0$.\\
    Denoting the radius of the $m^{\text{th}}$ order of the ring by $r_m$, the following can be proved after a bit of work:
    \begin{align*}
        r_m &= \sqrt{m\lambda R}\\
        r_m &= \sqrt{\left(m+\frac{1}{2}\right)\lambda R}
    \end{align*}
    for the \textbf{dark} and \textbf{bright} rings respectively. In the above results, $R$ is the radius of the spherical surface and $t$ is the thickness of the film\\
    Diameter of $m^{\text{th}}$ order ring $D_m$ = $\sqrt{4m\lambda R}$
    \end{minipage}
};

\node[fancytitle, right=10pt] at (box.north west) {Newton's rings};
\end{tikzpicture}

%Diffraction grating
\begin{tikzpicture}
\node [mybox] (box){%
    \begin{minipage}{0.3\textwidth}
        $a$ = $(n-1)\Delta$\\
    	\textbf{Single slit diffraction}:\\
    	If the rays make an angle $\theta$ with the normal, the corresponding phase difference $\phi$ is given as $\phi = \frac{2m}{\lambda}\Delta \sin{\theta}$, $\Delta$ being the distance b/w consecutive points \\
    	Intensity distribution: $I = I_0\frac{\sin^2{\beta}}{\beta^2}$ where $I_0$ is the intensity at $\theta = 0$ and $\beta = \frac{\pi a\sin{\theta}}{\lambda}$ \\
    	Minima: $a\sin{\theta} = m\lambda (m\neq 0)$\\
    	Maxima: $\tan{\beta} = \beta$\\
    	\textbf{Double slit diffraction}:\\
    	Similar as above, $\phi_1 = \frac{2\pi}{\lambda}d\sin{\theta}$\\
    	Intensity distribution: $I = 4I_0\frac{\sin^2{\beta}}{\beta^2}\cos^2{\gamma}$ where $\gamma = \frac{\phi_1}{2} = \frac{\pi}{\lambda}d\sin{\theta}$\\
    	Minima: $\gamma=(2n+1)\frac{\pi}{2}\implies a\sin{\theta} = m\lambda$ where $m=(1,2,3\ldots$)\\
    	Maxima: $\gamma=n\pi \implies d\sin{\theta}=n\lambda$ where $n=(0,1,2,\ldots)$\\
    	In all of the equations above, $d$ represents the \textbf{distance} between the two point sources and $a$ denotes the \textbf{width} of two parallel slits
    \end{minipage}
};

\node[fancytitle, right=10pt] at (box.north west) {Diffraction grating};
\end{tikzpicture}

\end{multicols*}
\end{document}
