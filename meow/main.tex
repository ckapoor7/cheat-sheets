\documentclass[a4paper]{article} 
\addtolength{\hoffset}{-2.25cm}
\addtolength{\textwidth}{4.5cm}
\addtolength{\voffset}{-3.25cm}
\addtolength{\textheight}{5cm}
\setlength{\parskip}{0pt}
\setlength{\parindent}{0in}

%	PACKAGES AND OTHER DOCUMENT CONFIGURATIONS


\usepackage[utf8]{inputenc} % Use UTF-8 encoding
\usepackage{microtype} % Slightly tweak font spacing for aesthetics
\usepackage[english]{babel}% Language hyphenation and typographical rules
\usepackage{url}
\usepackage{wrapfig}

\usepackage{circuitikz}
\usepackage{amsthm, amsmath, amssymb} % Mathematical typesetting
\usepackage{float} % Improved interface for floating objects
\usepackage[final, colorlinks = true, 
            linkcolor = black, 
            citecolor = black]{hyperref} % For hyperlinks in the PDF
\usepackage{graphicx, multicol} % Enhanced support for graphics
\usepackage{xcolor} % Driver-independent color extensions
\usepackage{tikz-qtree} % Easy tree drawing tool
\tikzset{every tree node/.style={align=center,anchor=north},
         level distance=2cm} % Configuration for q-trees
\usepackage{csquotes} % Context sensitive quotation facilities
\usepackage[yyyymmdd]{datetime} % Uses YEAR-MONTH-DAY format for dates
\renewcommand{\dateseparator}{-} % Sets dateseparator to '-'
\usepackage{fancyhdr} % Headers and footers
\pagestyle{fancy} % All pages have headers and footers
\fancyhead{}\renewcommand{\headrulewidth}{0pt} % Blank out the default header
\fancyfoot[L]{} % Custom footer text
\fancyfoot[C]{} % Custom footer text
\fancyfoot[R]{\thepage} % Custom footer text
\newcommand{\note}[1]{\marginpar{\scriptsize \textcolor{red}{#1}}} %

\begin{document}

%Title section

\fancyhead[C]{}
\hrule \medskip % Upper rule
\begin{minipage}{0.295\textwidth} 
\raggedright
\footnotesize
Chaitanya Kapoor \hfill\\   
2020A3PS1219P\hfill\\
f20201219@pilani.bits-pilani.ac.in
\end{minipage}
\begin{minipage}{0.4\textwidth} 
\centering 
\Large
Oscillations 
\end{minipage}
\begin{minipage}{0.295\textwidth} 
\raggedleft
\today\hfill\\
\end{minipage}
\medskip\hrule 
\bigskip

\section{Simple harmonic oscillators}
\begin{itemize}
    \item \underline{Simple pendulum}: 
    \begin{itemize}
        \item Differential equation $\rightarrow \ddot{\theta} + \frac{g}{L}\theta = 0$.   
        \item Time period $\rightarrow 2\pi\sqrt{\frac{L}{g}}$
    \end{itemize}
    \item \underline{Physical pendulum}:
    \begin{itemize}
        \item Differential equation $\rightarrow \ddot{\theta} + \frac{mgL}{2I}\theta = 0$.  
        \item Time period $\rightarrow 2\pi\sqrt{\frac{I_a}{mgL}}$\\ 
        Where $L$ is the distance between the axis of rotation and the center of mass of the body, and $I_a$ is moment of inertia about the \textbf{axis of rotation}. This equation can also be rewritten in terms of the radius of gyration of the body $k$:
        \begin{equation*}
            T = 2\pi\sqrt{\frac{(k^2+L^2)}{gL}}
        \end{equation*}
    \end{itemize}
    \item \underline{Spring-mass system}: 
    \begin{itemize}
        \item Equation of motion $\rightarrow x(t) = A\cos{\omega t}$ and $v(t) = -A\omega\sin{\omega t}$
        \item Kinetic energy $= \frac{1}{2}mv^2 = \frac{1}{2}m\omega^2A^2\sin^2{\omega t} = \frac{1}{2}m\omega^2A^2(1-\cos^2{\omega t}) = \frac{1}{2}m\omega^2(A^2-x^2)$ 
        \item Potential energy = $\frac{1}{2}kx^2$
        \item Total energy = $\frac{1}{2}m\omega^2A^2$
        
    \end{itemize}
\end{itemize}

\section{Damped harmonic motion}
Generally, in addition to the spring force $kx$, we must take into account a \textbf{damping force} which is proportional to the velocity of the oscillator, ie- $F_{\text{damp}} = -b\vec{v} = -b\dot{x}$. Hence, the equation of motion of the system takes the form
\begin{align*}
    m\ddot{x} = -kx-b\dot{x}\\
    \ddot{x}+\frac{b}{m}\dot{x}+\frac{k}{m}x = 0\\
    \therefore \ddot{x} + \gamma\dot{x} + \omega_0^2 x = 0
\end{align*}
Where $\gamma = \frac{b}{m}$ and $\omega_0 = \sqrt{\frac{k}{m}}$. The solution of this characteristic differential equation is 
\begin{align*}
    &x(t) = e^{-\frac{\gamma}{2}t}(C_1e^{qt} + C_2e^{-qt})\\
    &\text{Where } q = \sqrt{\frac{\gamma^2}{4}-\omega_0^2} = \omega
\end{align*}
Depending on the relation between $\gamma$ and $\omega_0$, we characterize the types of damped motion. 
\begin{enumerate}
    \item Over-damped : $\frac{\gamma^2}{4} > \omega_0$ and $x(t) = e^{-\frac{\gamma}{2}t}(C_1e^{qt} + C_2e^{-qt})$ 
    \item Critically damped : $\frac{\gamma^2}{4} = \omega_0$ and $x(t) = (C_1+C_2t)e^{-\frac{\gamma}{2}t}$
    \item Over damped : $\frac{\gamma^2}{4} < \omega_0$ and $x(t) = Ae^{-\frac{\gamma}{2}t}\cos{qt}$
\end{enumerate}
The energy of a damped system also decays with time, and can be described with the help of the parameter $\gamma$, ie: $\boxed{E = E_0e^{-\gamma t}}$. In this regard, a new parameter called the \textbf{Quality factor (Q-value)} is introduced. It can be defined in any two of the following ways
\begin{align*}
    Q &= \frac{\omega}{\gamma}\\
      &= 2\pi\left(\frac{\text{Energy at }t}{\text{Energy loss per cycle}}\right)
\end{align*}

\section{Forced oscillations}
A free or damped oscillation with a harmonic driving force $(F = F_0\cos{\omega t})$ has the differential equation 
\begin{equation*}
    \ddot{x} + \gamma\dot{x} + \omega_0^2 x = \frac{F_0}{m}\cos{\omega t}
\end{equation*}
We try a solution of the form $x=Ae^{i(\omega t-\delta)}$. After a series of steps, we arrive at the following expressions:
\begin{align*}
    A(\omega) &= \frac{F_0}{m\sqrt{(\omega_0^2-\omega^2)^2+\gamma^2\omega^2}}\\
   \delta(\omega) &= \arctan{\left(\frac{\gamma\omega}{\omega_0^2-\omega^2}\right)} 
\end{align*}
The cases of \textbf{resonant frequency} are of particular interest to us (when $\omega = \omega_0$). At this condition, 
\begin{align*}
    A(\omega_0) = A_0 &= \frac{F_0Q}{m\omega^2}\\
    \tan{\delta(\omega_0)} &= \infty
\end{align*}
Another case of interest is when the angular frequency is \textit{just less than} the resonant frequency. This is known as \textbf{amplitude resonance} and the particular frequency is denoted by $\omega_m$. The frequency and amplitude respectively are:
\begin{align*}
    \omega_m = \omega_0\sqrt{\left(1-\frac{1}{2Q^2}\right)} \approx \omega_0\left(1-\frac{1}{4Q^2}\right)\\
    A(\omega_m) = \frac{F_0}{m}\left(\frac{Q}{\sqrt{1-\frac{1}{4Q^2}}}\right)
\end{align*}
Next, we take into consideration a \textbf{forced oscillation without the action of a damping force}. The equation of motion for such a body is
\begin{equation*}
    \ddot{x} + \omega_0^2x = \frac{F_0}{m}\cos{\omega t}
\end{equation*}
Which has the solution
\begin{equation*}
    x(t) = A_0\cos{(\omega t+\varphi)} + \frac{F_0}{m(\omega^2-\omega_0^2)}\cos{\omega t}
\end{equation*}
All the symbols having their usual meaning.\\
A set of results for terms relating to power are listed below:
\begin{enumerate}
    \item \textbf{Average power} $(\bar{P})$ = $\frac{F_0}{2m\omega_0 Q}\frac{1}{\left(\frac{\omega_0}{\omega}-\frac{\omega}{\omega_0}\right)^2 + \frac{1}{Q^2}}$
    \item \textbf{Maximum power} $(\bar{P}_{\text{max}}) = \frac{F_0^2Q}{2m\omega_0}$
    \item \textbf{Half power frequency} $\omega = \omega_0\left(1\pm\frac{\omega_0}{2Q}\right) = \omega_0\left(1\pm\frac{\gamma}{2}\right) $
    \item \textbf{Bandwidth} $(\Delta\omega) = \gamma = \frac{\omega_0}{Q}$ which represents the \textbf{full width at half maximum} (FWHM)
\end{enumerate}
\end{document}
